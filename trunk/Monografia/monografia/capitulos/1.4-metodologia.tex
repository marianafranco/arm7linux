% Metodologia de Trabalho
\section{Metodologia de Trabalho} \label{sec:intro_metodologia}
\index{introducao!metodologia}

Para a realiza��o desse projeto de formatura procurou-se seguir uma metodologia de trabalho cuja as etapas s�o descritas a seguir:

\begin{itemize}

\item Estudo da Arquitetura ARM e da Placa Did�tica Evaluator-7T:

Antes de especificar as funcionalidades que seriam desenvolvidas, um estudo aprofundado da arquitetura ARM foi realizado para compreender o funcionamento do processador para o qual o microkernel foi desenvolvido, o  ARM7TDMI.

Al�m disso, rodamos alguns exemplos na placa did�tica Evaluator-7T para adquirir conhecimentos sobre o seu funcionamento e limita��es.


\item Montagem do Ambiente de Trabalho:

Paralelamente ao estudo descrito no item anterior, montamos um ambiente de trabalho utilizando a IDE CodeWarrior para o desenvolvimento do c�digo-fonte e o AXD Debugger para depurar o funcionamento do microkernel com ou sem a utiliza��o da placa did�tica.

Um reposit�rio de controle de vers�o tamb�m foi montado para estocar o material produzido durante do projeto (documenta��o e c�digo-fonte) e para sincronizar o trabalho dos integrantes do grupo.


\item Especifica��o Funcional do Microkernel:

O microkernel desenvolvido foi especificado nessa etapa, onde levantamos as funcionalidades b�sicas de um sistema operacional que deveriam ser implementadas, como o chaveamento de processos e as chamadas de sistema.


\item Desenvolvimento do Microkernel:

Nessa fase, foi desenvolvido o microkernel utilizando como base a especifica��o definida no item anterior.


\item An�lise do Microkernel e Conclus�es:

Ao final do desenvolvimento, com base nas dificuldades e solu��es encontradas, foi feita uma an�lise e conclus�o sobre o microkernel desenvolvido e sua poss�vel utiliza��o no Laborat�rio de Processadores para exemplificar os conceitos visto na disciplina de Sistemas Operacionais.


\end{itemize}
