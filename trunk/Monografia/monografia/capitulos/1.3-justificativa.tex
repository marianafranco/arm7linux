% Justificativa
\section{Justificativa}  \label{sec:intro_justificativa}
\index{introducao!justificativa}

O objetivo inicial do projeto era portar um sistema operacional Unix j� existente para a placa did�tica Evaluator-7T. 

Inicialmente pensamos em utilizar os sistemas Android e Minix 3, mas ao estudar o \emph{kernel} dos dois sistemas, vimos que os recursos de mem�ria necess�rios para execut�-los era muito maior que os 512kB dispon�veis na placa. Al�m disso, no caso do Minix 3, ter�amos que reescrever o \emph{assembly} do \emph{kernel} que atualmente s� tem vers�o para i386, para \emph{assembly} ARM, o que seria imposs�vel com o tempo dispon�vel para o projeto. Tamb�m foram realizadas pesquisas sobre outros Sistemas Operacionais para sistemas embarcados, cujos resultados e conclus�es encontram-se no ap�ndice \ref{chap:pesquisas}.

Assim surgiu a id�ia de desenvolver um \emph{microkernel} pr�prio, com as funcionalidades b�sicas de um sistema operacional, e que fosse de f�cil entendimento; pois como mencionado anteriormente, espera-se que o material desenvolvido seja destinado a melhorar e aproximar o ensino de Sistemas Operacionais com as experi�ncias do Laborat�rio de Microprocessadores.