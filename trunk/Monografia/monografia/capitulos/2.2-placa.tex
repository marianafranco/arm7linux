
% A Placa Experimental Evaluator-7T
\section{A Placa Experimental Evaluator-7T} \label{sec:conceitos_placa}
\index{conceitos!placa}

O principal elemento de hardware deste projeto é a placa experimental ARM Evaluator-7T, baseada no processador ARM7TDMI, um processador RISC de 32 bits capaz de executar o conjunto de instruções denominado Thumb.

Os elementos presentes na arquitetura da placa Evaluator-7T são os seguintes:

\begin{arquitetura_eval7t}

\item Microcontrolador Samsung KS32C50100

\item 512kB EPROM flash

\item 512kB RAM estática (SRAM)

\item Dois conectores RS232 de 9 pinos tipo D

\item Botões de reset e de interrupção

\item Quatro LEDs programáveis pelo usuário e um display de 7 segmentos

\item Entrada de usuário por um interruptor DIP com 4 elementos

\item Conector Multi-ICE

\item Clock de 10MHz (o processador usa-o para gerar um clock de 50MHz)

\item Regulador de tensão de 3.3V

\end{arquitetura_eval_7t}

%inserir diagrama da arquitetura da placa aqui

Com relação à memória flash da placa, ela vem de fábrica com o bootstrap loader da placa e programa monitor de debug. O restante dela pode ser usado para os programas de usuário.

Já em relação às duas portas seriais presentes na placa, cada uma tem usos específicos. A primeira, chamada DEBUG, é usada pelo monitor de debug ou pelo programa bootstrap presente na placa. Ela está conectada ao UART1 do microcontrolador. A segunda, chamada USER, é de uso genérico e está disponível para uso em programas. Ela está conectada ao UART0 do microcontrolador.