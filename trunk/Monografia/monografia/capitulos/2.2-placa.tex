 % A Placa Experimental Evaluator-7T
\section{A Placa Experimental Evaluator-7T} \label{sec:conceitos_placa}
\index{conceitos!placa}

O principal elemento de hardware deste projeto � a placa experimental ARM Evaluator-7T, baseada no processador ARM7TDMI, um processador RISC de 32 bits capaz de executar o conjunto de instru��es denominado Thumb.

Os elementos presentes na arquitetura da placa Evaluator-7T s�o os seguintes:

\begin{itemize}

\item Microcontrolador Samsung KS32C50100

\item 512kB EPROM flash

\item 512kB RAM est�tica (SRAM)

\item Dois conectores RS232 de 9 pinos tipo D

\item Bot�es de reset e de interrup��o

\item Quatro LEDs program�veis pelo usu�rio e um display de 7 segmentos

\item Entrada de usu�rio por um interruptor DIP com 4 elementos

\item Conector Multi-ICE

\item Clock de 10MHz (o processador usa-o para gerar um clock de 50MHz)

\item Regulador de tens�o de 3.3V

\end{itemize}

%inserir diagrama da arquitetura da placa aqui

Com rela��o � mem�ria flash da placa, ela vem de f�brica com o bootstrap loader da placa e programa monitor de debug. O restante dela pode ser usado para os programas de usu�rio.

J� em rela��o �s duas portas seriais presentes na placa, cada uma tem usos espec�ficos. A primeira, chamada DEBUG, � usada pelo monitor de debug ou pelo programa bootstrap presente na placa. Ela est� conectada ao UART1 do microcontrolador. A segunda, chamada USER, � de uso gen�rico e est� dispon�vel para uso em programas. Ela est� conectada ao UART0 do microcontrolador.