\section{\emph{Threads}} \label{cap:processos}

O kernel pode lidar com no m�ximo nove \emph{threads}, nomeados de task1 a task9 no arquivo tasks.c. Como eles n�o t�m �rea de dados pr�pria, n�o pode-se cham�-los de processos. A implica��o de se ter uma �rea de dados em comum � que todos os processos que rodam um mesmo programa compartilham os valores das vari�veis globais, mas n�o locais. O mais correto, portanto, seria o cham�-las de \emph{threads}.

Foram criados alguns programas exemplo que se utilizam dos perif�ricos da placa. Eles testam:

\begin{itemize}
\item{O chaveamento de processos}
\item{Os diversos perif�ricos da placa: LEDs, display de 7 segmentos, chave DIPs e o bot�o}
\item{Comunica��o da placa com um terminal}
\item{A duplica��o, cria��o e morte de um processo}
\item{Rotina de exclus�o m�tua}
\end{itemize}