% Motiva��o
\section{Motiva��o} \label{sec:intro_motivacao}
\index{introducao!motivacao}

As disciplinas de Laborat�rio de Processadores e de Sistemas Operacionais do curso de Engenharia da Computa��o na Escola Polit�cnica da USP, atualmente, est�o muito distantes entre si, no entanto o conte�do das mesmas � muito pr�ximo.

Pensando ent�o em como aproximar essas duas disciplinas, surgiu a id�ia de desenvolver uma ferramenta did�tica que unisse um \emph{hardware} de estudo simples a um sistema operacional igualmente simples, e que pudesse ser utilizada nas experi�ncias do Laborat�rio de Processadores.

Para cria��o desta ferramenta, foi escolhida a placa experimental ARM Evaluator-7T, que possui uma arquitetura ARM e um poder de processamento bastante superior aos sistemas did�ticos utilizados atualmente (baseados nos processadores Intel 8051 e o Motorola 68000). Assim sendo, pretende-se atualizar o material did�tico da disciplina de processadores, trazendo um sistema mais moderno e mais pr�ximo da realidade atual, al�m de poder se relacionar com o conte�do da disciplina de Sistemas Operacionais.

Outra motiva��o do projeto foi aprofundar nossos conhecimentos sobre sistemas operacionais e sobre a arquitetura dos processadores ARM, visto que este processador �, hoje em dia, largamente utilizado em sistemas embarcados e aparelhos celulares.
