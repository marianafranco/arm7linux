
\newcommand{\moisem}[0]{\href{http://www.lti.pcs.usp.br/moise}{$\mathcal{M}$\textsc{oise}$^{+}$}}


\chapter{Exemplo de especifica��o organizacional no formato XML}
\label{anx:exemploOS}

\section{Exemplo da escola}

Este anexo � a especifica��o organizacional em formato XML (conforme
utilizado na implementa��o do \moisem) para o exemplo da escola que
foi desenvolvido no \prettyref{chp:moisem}.

\begin{espacosimples}
\begin{verbatim}
<?xml version="1.0" encoding="UTF-8"?>
<?xml-stylesheet href="xml/os.xsl" type="text/xsl" ?>
<!DOCTYPE OrganizationalSpecification SYSTEM "os.dtd">

<OrganizationalSpecification id="joj">
  <StructuralSpecification>
    <RolesDefinition>
      <Role id="docente">    <extends role="soc" /> </Role>
      <Role id="aluno">      <extends role="soc" />  </Role>
      <Role id="diretor">    <extends role="docente" />  </Role>
      <Role id="professor">  <extends role="docente" />  </Role>
    </RolesDefinition>
    <LinksType>
      <LinkType id="acquaintance" />
      <LinkType id="communication" />
      <LinkType id="authority" />
    </LinksType>

    <GroupSpecification id="escola">

o que segue nao faz sentido :-)

<OrganizationalSpecification id="joj">
  <StructuralSpecification>
    <RolesDefinition>
      <Role id="docente">    <extends role="soc" /> </Role>
      <Role id="aluno">      <extends role="soc" />  </Role>
      <Role id="diretor">    <extends role="docente" />  </Role>
      <Role id="professor">  <extends role="docente" />  </Role>
    </RolesDefinition>
    <LinksType>
      <LinkType id="acquaintance" />
      <LinkType id="communication" />
      <LinkType id="authority" />
    </LinksType>

\end{verbatim}
\end{espacosimples}

bla bla bla \\
bla bla bla \\
bla bla bla \\
bla bla bla \\
bla bla bla \\
bla bla bla \\
bla bla bla \\
bla bla bla \\
bla bla bla \\
bla bla bla \\
bla bla bla \\
bla bla bla \\
bla bla bla \\
bla bla bla \\
bla bla bla \\
bla bla bla \\
bla bla bla \\
bla bla bla \\
bla bla bla \\
bla bla bla \\
bla bla bla \\
bla bla bla \\
bla bla bla \\
bla bla bla \\
bla bla bla \\
bla bla bla \\
bla bla bla \\
bla bla bla \\
bla bla bla \\
bla bla bla \\
bla bla bla \\
bla bla bla \\
bla bla bla \\
bla bla bla \\
bla bla bla \\
bla bla bla \\
bla bla bla \\
bla bla bla \\
bla bla bla \\
bla bla bla \\
bla bla bla \\
bla bla bla \\
bla bla bla \\
bla bla bla \\
bla bla bla \\
bla bla bla \\
bla bla bla \\
bla bla bla \\
bla bla bla \\
bla bla bla \\
bla bla bla \\
bla bla bla \\
bla bla bla \\
bla bla bla \\
bla bla bla \\
bla bla bla \\
bla bla bla \\
bla bla bla \\
bla bla bla \\
bla bla bla \\
bla bla bla \\
bla bla bla \\
bla bla bla \\
bla bla bla \\
bla bla bla \\
bla bla bla \\
bla bla bla \\
bla bla bla \\
bla bla bla \\
bla bla bla \\
bla bla bla \\
bla bla bla \\


\chapter{Um novo anexo}

bla bla bla \\
bla bla bla \\
bla bla bla \\
bla bla bla \\
bla bla bla \\
bla bla bla \\
bla bla bla \\
bla bla bla \\
bla bla bla \\
bla bla bla \\
bla bla bla \\
bla bla bla \\
bla bla bla \\
bla bla bla \\
bla bla bla \\
bla bla bla \\
bla bla bla \\
bla bla bla \\
bla bla bla \\
bla bla bla \\
bla bla bla \\
bla bla bla \\
bla bla bla \\
bla bla bla \\
bla bla bla \\
bla bla bla \\
bla bla bla \\
bla bla bla \\
bla bla bla \\
bla bla bla \\
bla bla bla \\
bla bla bla \\
bla bla bla \\
bla bla bla \\
bla bla bla \\
bla bla bla \\
bla bla bla \\
bla bla bla \\
bla bla bla \\
bla bla bla \\
bla bla bla \\
bla bla bla \\
bla bla bla \\
bla bla bla \\
bla bla bla \\
bla bla bla \\
bla bla bla \\
bla bla bla \\
bla bla bla \\
bla bla bla \\
bla bla bla \\
bla bla bla \\
bla bla bla \\
bla bla bla \\
bla bla bla \\
bla bla bla \\
bla bla bla \\
bla bla bla \\
bla bla bla \\
bla bla bla \\
bla bla bla \\
bla bla bla \\
bla bla bla \\


%%% Local Variables: 
%%% mode: latex
%%% TeX-master: "tese"
%%% x-symbol-8bits: t 
%%% End: 
