% Organiza��o do Trabalho
\section{Organiza��o do Documento} \label{sec:intro_organizacao}
\index{introducao!organizacao}

Este documento foi estruturado da seguinte maneira:

\begin{itemize}

\item Cap�tulo 1 (Introdu��o):

Apresenta objetivo, motiva��es, justificativas e a metodologia do trabalho.


\item Cap�tulo 2 (Conceitos e Tecnologias Envolvidas):

Contextualiza o leitor em aspectos t�cnicos espec�ficos utilizados no desenvolvimento do trabalho.


\item Cap�tulo 3 (O Sistema Operacional KinOS):

Descreve como o \emph{microkernel} foi desenvolvido, quais as suas funcionalidades e como funciona a sua integra��o com os perif�ricos da placa did�tica, com o terminal e com os outros programas implementados.


\item Cap�tulo 4 (Considera��es Finais):

Analisa os resultados obtidos em rela��o ao objetivo do projeto, as conclus�es, as contribui��es deste trabalho e indica poss�veis trabalhos futuros com base neste.


\end{itemize}