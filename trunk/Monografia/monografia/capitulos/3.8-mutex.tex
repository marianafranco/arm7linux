\section{\emph{Mutex}}

O \emph{mutex} ou exclus�o m�tua, � uma t�cnica usada para evitar que dois processos tenham acesso a um mesmo espa�o de mem�ria. Seu funcionamento � baseado em uma vari�vel que pode ter apenas dois valores, 0 ou 1. Caso ela seja 0, ela indica que a �rea cr�tica pode ser acessada e 1 caso contr�rio. No caso de um processo n�o obter acesso, ele fica em espera ativa, at� que o processo que o bloqueou o desfa�a.

No exemplo que � dado no KinOS, tem-se duas fun��es que realizam o travamento e o destravamento do \emph{mutex} e a vari�vel semaphore, que guarda o valor do \emph{mutex}. A primeira fun��o se chama \verb|mutex_gatelock|, que pode ser observada abaixo.

\begin{lstlisting}
void mutex_gatelock (void) {
	__asm {
		spin:
		mov		r1, &semaphore
		mov		r2, #1
		swp		r3,r2,[r1]
		cmp		r3,#1
		beq		spin
	}
}
\end{lstlisting}

r1 recebe o endere�o da vari�vel semaphore, e r2 recebe 1. A fun��o at�mica swp � que permite o correto funcionamento do \emph{mutex}: em uma instru��o indivis�vel, o conte�do de semaphore � colocado em r3, e 1 � colocado em semaphore. Com isso, � imposs�vel que haja uma interrup��o entre estas duas a��es, o que poderia arruinar uma rotina de \emph{mutex}. Finalmente, caso semaphore j� estivesse ativo quando chamado, a rotina seria executada novamente.

\begin{lstlisting}
void mutex_gateunlock (void)  {
	__asm  {
		mov		r1, &semaphore
		mov		r2, #0
		swp		r0,r2,[r1]
	}
}
\end{lstlisting}

O destravamento � feito de modo similar, com a mesma instru��o. S� que neste caso, o valor 0 � colocado em semaphore.

Por�m, as rotinas apresentadas n�o s�o chamadas diretamente. Usa-se as chamadas abaixo.

\begin{lstlisting}
#define WAIT 		while (semaphore==1) {} mutex_gatelock(); 
#define SIGNAL 		mutex_gateunlock(); 	
\end{lstlisting}

O motivo � que ao se usar o \verb|WAIT| quando o \emph{mutex} est� ativo, faz com que o programa entre em uma espera ativa.

\paragraph{}