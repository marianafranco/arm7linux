\section{Inspira��o}

Grande parte do c�digo foi baseada do c�digo presente nos exemplos inclu�dos no CD de demonstra��o da placa, desenvolvido por Andrew N. Sloss. O principal deles, � o c�digo \emph{mutex}, de onde foi baseado o chaveamento de processos, a fun��o de mutex e as rotinas de manipula��o de hardware. Eis a inspira��o de cada uma das partes do projeto:

\paragraph{Chaveamento de processos} O chaveamento de processos original, contido no projeto mutex era de apenas duas threads. Modifica��es foram realizadas para fazer com que o n�mero de processos passasse de duas para nove.

\paragraph{Inicializa��o} A inicializa��o foi baseada no c�digo do mutex, mas grandes altera��es foram feitas e pouqu�ssimas coisas ainda restam do c�digo original

\paragraph{Shell} O c�digo foi baseado no exemplo da porta serial e tamb�m no Sistema Operacional ISOS \cite{ISOSPage}.

\paragraph{Interrup��o de hardware} O c�digo foi baseado no exemplo do mutex, mas grandes altera��es foram realizadas

\paragraph{Interrup��o de software} Foi baseado no exemplo SWI, tamb�m contido no CD da placa

\paragraph{Rotinas de manipula��o de perif�ricos} As rotinas foram praticamente copiadas do c�digo do mutex, que utiliza todas elas

\paragraph{Chamadas de sistema} Todas as chamadas de sistema foram inteiramente desenvolvidas durante o projeto

\paragraph{Mutex} O c�digo do mutex foi inteiramente copiado do exemplo com o mesmo nome
