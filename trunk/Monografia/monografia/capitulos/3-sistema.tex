%
%~~
%~~  O Sistema Operacional KinOS
%~~
%

\chapter{O Sistema Operacional KinOS} \label{chap:sistema}
\index{sistema}


Quando um sistema operacional � projetado, geralmente � adotada uma arquitetura em m�dulos, a fim de se dividir esta tarefa complexa em v�rias partes com fun��es bem definidas, como o gerenciamento de E/S, drivers e aplica��es. Nesta arquitetura, o m�dulo que � utilizado por todos os outros e que interage diretamente com o hardware � chamado de kernel. Por seu contato direto com o hardware, este c�digo � escrito em assembly, e por  ser utilizado por todos os outros m�dulos do sistema operacional, deve ser r�pido para n�o comprometer a performance do sistema.  Ele roda quando o hardware � ligado, a fim de se inicializar o sistema, e ap�s isso apenas quando h� algum tipo de interrup��o.

\section{Tipos de kernel}

H� basicamente dois tipos de kernel: monol�ticos e microkernels.
