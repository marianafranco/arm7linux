
% Acr{\^o}nimos (siglas)
% uso: \acro{sigla}{descri{\c c}{\~a}o} cria um acr{\^o}nimo e imprime ele num ambiente \begin{acronym}
%      \acrodef{sigla}{descri{\c c}{\~a}o} cria um acr{\^o}nimo mas n{\~a}o imprime ele
%      \acf{sigla} escreve a por extenso e depois a (sigla)
%      \acl{sigla} escreve a por extenso
%      \ac{sigla} coloca a descri{\c c}{\~a}o da sigla se for sua primeira apari{\c c}{\~a}o

\pretextualchapter{Lista de Abreviaturas}% e Siglas}
\begin{acronym}
\acro{ADS}{ARM Developer Suite}
\acro{ALU}{Arithmetic Logic Unit}
\acro{ARM}{Advanced RISC Machine}
\acro{CISC}{Complex Instruction Set Computer}
\acro{CPSR}{Current Program Status Register}
\acro{FIQ}{Fast Interrupt}
\acro{IDE}{Integrated Development Environment}
\acro{IRQ}{Interrupt Request}
\acro{LR}{Link Register}
\acro{PC}{Program Counter}
\acro{PSR}{Program Status Register}
\acro{RISC}{Reduced Instruction Set Computer}
\acro{SP}{Stack Pointer}
\acro{SPRS}{Saved Program Register}
\acro{SWI}{Software Interruption}
\acro{USP}{Universidade de S�o Paulo}

%\acro{B2B}{Business to Business}
%\acro{ED}{Especifica{\c c}{\~a}o De{\^o}ntica}
%\acro{EE}{Especifica{\c c}{\~a}o Estrutural}
%\acro{EF}{Especifica{\c c}{\~a}o Funcional}
%\acro{EnO}{Entidade Organizacional}
%\acro{EO}{Especifica{\c c}{\~a}o Organizacional}
%\acro{ES}{Esquema Social}
%\acro{IA}{Intelig{\^e}ncia Artificial}
%\acro{IAD}{Intelig{\^e}ncia Artificial Distribu{\'\i}da}
%\acro{KQML}{Knowledge Query and Manipulation Language}
%\acro{MOISE}{Model of Organization for multI-agent SystEms}
%\acro{OO}{Orienta{\c c}{\~a}o a Objetos}
%\acro{RDP}{Resolu{\c c}{\~a}o Distribu{\'\i}da de Problemas}
%\acro{SMA}{Sistemas Multiagentes}
%\acro{TAEMS}{Task Analysis, Environment Modeling, and Simulation}
\end{acronym}
 
%%% Local Variables: 
%%% mode: latex
%%% TeX-master: "tese"
%%% End: 
