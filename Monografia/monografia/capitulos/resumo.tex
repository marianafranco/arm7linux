\begin{resumo}   
	
O uso de dispositivos m�veis como celulares, \emph{smartphones}, tocadores de MP3 e \emph{video-games} port�teis � cada vez mais comum. A atual l�der no segmento de processadores de baixa pot�ncia, essencial nestes tipos de aparelhos, � a empresa inglesa ARM. � fim de se modernizar o equipamento usado em aulas, ela disponibilizou � Escola Polit�cnica algumas placas ARM Evaluator 7-T, cujo processador, o ARM7TDMI, � usado em eletr�nicos muito populares atualmente, como o Apple iPod e o Nintendo DS. 

Assim, utilizando-se desse hardware mais moderno e pensando em aproximar o ensino das disciplinas de Sistemas Operacionais e do Laborat�rio de Microprocessadores, este projeto visa o desenvolvimento de um \emph{microkernel} para a placa ARM Evaluator-7T, tema que engloba conhecimento de ambas as disciplinas. 

O \emph{microkernel} desenvolvido, chamado de KinOS, � provido de algumas fun��es b�sicas, e � apenas o primeiro passo para um projeto muito maior, de desenvolvimento de um sistema operacional totalmente feito por alunos da Escola Polit�cnica. 

Dentre as fun��es b�sicas deste \emph{microkernel}, podemos citar o chaveamento de \emph{threads}, algumas chamadas de sistema (\emph{fork}, \emph{exec} e \emph{exit}), fun��es para manipula��o de perif�ricos e comunica��o atrav�s de um terminal.

\end{resumo}